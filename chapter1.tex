\section[Gesprächsergebnisse]{Zusammenfassung bisheriger Gesprächsergebnisse}

Die thandsome GbR verpflichtet sich, die thandsome Black Door auf Basis von zertifiziertem Holz herzustellen. 
Die Fertigung dauert voraussichtlich ca. 3 Monate nach verbindlichem Abschluss des Hauptvertrages, wobei die Lieferung 
im Mai 2024 erfolgen soll. Nordseeheilbad Carolinensiel-Harlesiel übernimmt die notwendigen 
Genehmigungen für die Installation und stellt den Transport sicher. Eine Bekundung zur Zusammenarbeit liegt vor, und Marcus Harazin 
arbeitet an den behördlichen Freigaben. 


\begin{itemize}
    \item 01/2024: Ausarbeiten und Unterzeichnen der Absichtserklärung \- LOI
    \item 02/2024: Ausarbeiten und Unterzeichnen des Vertrages 
    \item 03/2024 - 04/2024: Fertigung der thandsome Black Door 
    \item 05/2024: Lieferung der thandsome Black Door an Carolinensiel 
    \end{itemize}
    


\section[Konkretisierung]{Konkretisierung des Transaktionsvorhabens}

Wie Vertragsparteien verhandeln über den Abschluss eines Lizenzvertrages für die Nutzung der thandsome Black Door.
Bilder zur Veranschaulichung des Projektes sind hier in der Anlage "Projektabbildungen" beigefügt. 
Die Kosten für die thandsome Black Door werden überarbeitet, um den Einsatz von zertifiziertem
Holz zu berücksichtigen. Das erste Angebot für nicht zertifiziertes Holz betrug einmalig
4500 Euro sowie 200 Euro jährlich über 10 Jahre. 


\section[Geheimhaltungsverpflichtungt]{Geheimhaltungsverpflichtung}

Beide Parteien verpflichten sich zur Verschwiegenheit. 
Informationen und Kosten dürfen nicht an Dritte weitergegeben werden.
Im Falle einer Zuwiderhandlung gegen die Bestimmungen dieses Vertrags wird eine Konventionalstrafe fällig. 
Die Höhe der Konventionalstrafe beträgt das Doppelte des vereinbarten Preises für die Tür gemäß dieses Vertrags. 
Diese Konventionalstrafe dient als Schadensersatz und nicht als Strafe und schließt die Geltendmachung weiterer 
Schadensersatzansprüche aus.

  
\section[erhaltene Dokumente]{Herausgabe- bzw. Vernichtungsanspruch von erhaltenen Dokumenten}

Dokumente werden in einem Cloud-Bereich geteilt und nach Abschluss des Vertrages von beiden Seiten aus steuerlichen und Handelsrechtlichen Gründen 
10 Jahre verwahrt. Diese dürfen nicht mit Dritten geteilt werden. 

  
\section[Bindungswirkung]{Hinweis auf die fehlende Bindungswirkung des LOI}
 
Dieser LOI ist nicht bindend und stellt keine verbindliche Vereinbarung dar. 
Er dient als Grundlage für die weiteren Verhandlungen. 

  
\section[Beendigungsgründe]{Beendigungsgründe für die laufenden Verhandlungen}

Die Verhandlungen können beendet werden, wenn eine der Parteien wesentliche Bestimmungen
des Vertrages nicht erfüllt. Jede Partei hat das Recht, die Verhandlungen mit einer schriftlichen Mitteilung zu beenden. 
Etwaige Schadensersatzansprüche im Falle einer Kündigung bleiben unberührt.

  
\section[Exklusivitätsklausel]{Exklusivitätsklausel}
 
Die thandsome Black Door wird nicht exklusiv an Carolinensiel lizensiert.
Es werden mehrere Auflagen produziert, wobei Carolinensiel die limitierte Auflage mit der Nummer 1 erhält. 

\newpage
\section[Geheimhaltungsverpflichtung]{Geheimhaltungsverpflichtung}

Dieser LOI regelt die Absicht der Parteien zur Zusammenarbeit und hat nur Gültigkeit im Rahmen der weiteren Verhandlungen. Der LOI bildet keine rechtsverbindliche Vereinbarung. 
Wir sehen einer erfolgreichen Zusammenarbeit entgegen und werden die weiteren Schritte in Kürze einleiten. 


\vspace{2cm}
\begin{flushleft}
    \makebox[.4\textwidth]{\hrulefill}\hfill    \makebox[.4\textwidth]{\hrulefill}\\
    \makebox[.4\textwidth]{(Ort, Datum)}\hfill
    \makebox[.4\textwidth]{(Unterschrift, thandsome GbR)}\\
\end{flushleft}

\vspace{2cm}
\begin{flushleft}
    \makebox[.4\textwidth]{\hrulefill}\hfill    \makebox[.4\textwidth]{\hrulefill}\\
    \makebox[.4\textwidth]{(Ort, Datum)}\hfill
    \makebox[.4\textwidth]{(Unterschrift, Nordseeheilbad Carolinensiel-Harlesiel)}\\
\end{flushleft}